\documentclass{article}
\usepackage[utf8]{inputenc}
\usepackage[spanish]{babel}
\usepackage{listings}
\usepackage{graphicx}
\graphicspath{ {images/} }
\usepackage{cite}

\begin{document}

\begin{titlepage}
    \begin{center}
        \vspace*{1cm}
            
        \Huge
        \textbf{Proyecto Final}
            
        \vspace{0.5cm}
        \LARGE
        Los primeros pasos
            
        \vspace{1.5cm}
            
        \textbf{Mateo Alejandro Bravo Revelo \\ C.C. 1010156637}
            
        \vfill
            
        \vspace{0.8cm}
            
        \Large
        Despartamento de Ingeniería Electrónica y Telecomunicaciones\\
        Universidad de Antioquia\\
        Medellín\\
        Marzo de 2021
            
    \end{center}
\end{titlepage}

\tableofcontents

\newpage



\section{Ideación del proyecto final.} \label{contents}
El gusto por los dibujos animados antiguos me lleva a proponerme desarrollar un videojuego tipo \textit{side-scrolle} para el proyecto final de la materia Informática II. El género del videojuego será de plataformas en dos dimensiones, basado en la mecánica de videojuegos dispara y corre, donde el jugador podrá desplazarse de izquierda a derecha, con la posibilidad de saltar y también generar disparos a sus enemigos. \\\\
En la siguiente sección mostraré las ideas para el desarrollo del videojuego: \\\\
En el \textit{home} se mostrará una calle con edificaciones haciendo alusión a New York de los años 30. En esta calle el usuario tendrá la posibilidad de entrar a tres edificios. El primer edificio llevará al usuario  al videojuego principal. Al segundo edificio se entrará en un modo de juego \textit{Boss Battle}, donde se tendrá que derrotar al jefe, a este edificio se podrá entrar una vez haya terminado el juego del primer edificio. Cuando se haya derrotado al jefe acabará la historia del videojuego. Pero todo no acaba ahí, el usuario podrá entrar al tercer edificio que es un minijuego donde debe saltar a la cuerda. \\\\
En el primer edificio estará el juego principal. Ahí el usuario deberá lograr llegar al punto final en donde habrá una puerta con un mensaje de \textit{Finish}. En este momento acabará el juego principal. Pero para poder llegar hasta ese punto el usuario comenzará únicamente con tres vidas y deberá pasar los obstáculos que son generados por personajes. El juego principal habrá dos tipos de personajes. El primer tipo de personajes que son estáticos y lanzaran objetos que quitarán la vida al usuario, mientras que el otro tipo de personajes, algunos caerán del cielo y otros aparecerán en el camino tratando de acercase al usuario y así hacerle daño (quitarle vida), el objetivo es que el usuario pueda esquivarlos o darles de baja a los dos tipos de personajes. Para la defensa y ataque el usuario tendrá disparos infinitos. Además, en el camino el usuario tendrá que tomar unas monedas para poder jugar el minijuego (tercer edificio). Algo para tener en cuenta es que cuando el usuario muere (las tres vidas se terminan) automáticamente comenzara desde el inicio del juego. \\\\
En el segundo edificio el usuario tiene que derrotar al jefe del videojuego, cabe aclarar que el usuario puede entrar a este edificio únicamente cuando haya acabado el juego del primer edificio. El usuario entra con tres vidas a derrotar al jefe del videojuego, el jefe será un personaje  extremadamente grande y con muchos poderes, él lanzará muchos objetos característicos. El objetivo es que el usuario pueda derrotar al jefe esquivando los objetos que lanza y también ayudándose de los disparos, pero para esto el usuario tiene que tener una gran habilidad en los movimientos del videojuego y de su personaje. Cuando muere el usuario deberá iniciar la pelea con el jefe desde el principio (con la vida del jefe completa) en el segundo edificio. \\\\
En el tercer edificio hay un minijuego de saltar a la cuerda. Pero para entrar aquí el usuario debe tomar un número determinado de monedas que están en el primer edificio. La cuerda la mueven dos personajes y el usuario para salir victorioso debe saltar por un minuto (la velocidad de la cuerda aumentará con el tiempo), si la cuerda toca al usuario perderá el juego, pero puede intentarlo tantas veces quiera.\\\\
Todos los personajes y el ambiente estarán inspirados en la era dorada de la animación estadounidense en especial los dibujos animados de los años 30. Así como el soundtrack será jazz acorde a los movimientos de los obstáculos, villanos y del usuario. Los tipos de letra y colores de los menús de registro y de instrucciones serán inspirados en anuncios de esta época.\\\\








\end{document}
